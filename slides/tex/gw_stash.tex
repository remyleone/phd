% IEEE 802.15.4

\begin{frame}{Couches basses}

  \begin{block}{Objectifs}
    \begin{itemize}
      \item Fournir une abstraction du médium de communication
      \item Fournir des topologies de communications
    \end{itemize}
  \end{block}

  \begin{block}{Différents compromis}
    \begin{itemize}
      \item Long Range / Short Range
      \item Topologie étoilée / maillée / cluster
      \item Verticalement intégrée ou non
    \end{itemize}
  \end{block}

  \pnote{
    ---
    Il existe beaucoup de technologies standardisées ou non.
    Parler un peu de Zigbee, Z-wave, le bluetooth low energy.
    Dans un autre domaine Sigfox et Lora alliance
    ---
  }
\end{frame}

\begin{frame}{IEEE 802.15.4}

  \begin{block}{Points essentiels}
    \begin{itemize}
      \item Bande ISM
      \item Accès courte portée (10 - 50 mètres)
      \item Topologie étoilée, maillée
      \item 250 kb/s - 127 octets de charge utile
      \item Chaque noeud est identifié (EUI-64/EUI-16)
      \item Utilisée par d'autres standards (Zigbee, Thread)
    \end{itemize}
  \end{block}

  \begin{block}{Role de la passerelle}
    \begin{itemize}
      \item Coordinateur Personnal Area Network
      \item Fournit les identifiants courts
    \end{itemize}
  \end{block}

  \pnote{    
      - Justifications:
      - Faible cout de la radio et de la consommation
      - Utilisation de la bande ISM libre de droit (par contre la bande est chargée)
      - Full Function Device (peuvent communiquer entre eux) / Reduced Function Device (doit communiquer avec un FFD)
      - Utilisé par de nombreux autres standards (littérature abondante)
      - Mécanisme d'acquittement 
      - Comme la portée est limitée on va construire un réseau multi-sauts par dessus
      - Progrès récent pour faire du channel hopping (TSCH) et mitiger les problèmes de pertes
      ---
      - Le PAN fournit les identifiants courts pour prendre 16 bits au lieu de 64 dans chaque trame.
  }
\end{frame}

% 6LoWPAN

\begin{frame}{Couche Réseau}
  \begin{block}{Objectifs}
    \begin{itemize}
      \item Plan d'adressage
      \item Interconnexion avec l'existant
    \end{itemize}
  \end{block}

  \begin{block}{Contraintes}
    \begin{itemize}
      \item IPv6 non applicable directement
    \end{itemize}
  \end{block}

  \begin{block}{Role de la passerelle a cette couche}
    \begin{itemize}
      \item LBR
      \item Encapsulation / Compression
      \item Enregistrement des adresses
    \end{itemize}
  \end{block}

  \pnote{
    IPv6 définit une charge minimale supérieur à celle suportée par 15.4
  }

\end{frame}

% RPL

\begin{frame}{Routage dans les LLNs}
  \begin{block}{Contraintes}
    
  \end{block}
\end{frame}

\begin{frame}{RPL}
  \begin{block}{Point essentiels}
    \begin{itemize}
      \item Routage proactif
      \item Routes montantes / Routes descendantes
    \end{itemize}
  \end{block}

  \begin{block}{Role de la passerelle}
    \begin{itemize}
      \item Topologie de routage accessible dans certaines configurations
    \end{itemize}
  \end{block}
\end{frame}

% CoAP

\begin{frame}{Couche applicative}
  \begin{block}{Objectifs}
    \begin{itemize}
      \item Contient les messages applicatifs
      \item Communication M2M
    \end{itemize}
  \end{block}

  \pnote{

  }
\end{frame}

\begin{frame}{CoAP}
  \begin{block}{Points essentiels}
    \begin{itemize}
      \item Sans connexion (UDP)
      \item Architecture REST
    \end{itemize}
  \end{block}
  \begin{block}{Role de la passerelle}
    \begin{itemize}
      \item Mécanisme de proxy-inverse
    \end{itemize}
  \end{block}
\end{frame}
