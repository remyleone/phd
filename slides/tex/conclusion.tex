\section{Conclusion}

\begin{frame}\frametitle{Conclusion}

  \begin{alertblock}{Contributions}
    \begin{itemize}
      \item Les passerelles peuvent héberger des fonctionnalités réseau avancées
      \item La supervision passive peut fournir des estimations de consommation énergétique de manière transparente
      \item Un cache adaptatif peut adapter la quantité de requêtes entrantes dans un réseau contraint
      % \item La recherche reproductible et des cycles de développement courts sont possibles dans ce champ d'étude
    \end{itemize}
  \end{alertblock}

\end{frame}

\begin{frame}{Ouvertures}
  \begin{block}{Mesure implicite}
    \begin{itemize}
        \item Expérimentation sur une couche MAC plus fiable et déterministe (TSCH)
        \item Modèle de déduction plus fin des protocoles réseau
    \end{itemize}
  \end{block}

  \begin{block}{Optimisation des ressources}
    \begin{itemize}
      \item Maximisation de l'utilité du réseau sous plusieurs critères (délai, redonance, consommation énergétique)
      \item Reconfiguration dynamique
    \end{itemize}
  \end{block}

  \begin{alertblock}{Fonctionnalités supplémentaire des passerelles}
    \begin{itemize}
      \item Interopérabilité logicielle
      \item Ajout de services
      \item Aggrégation et découverte de services
    \end{itemize}
  \end{alertblock}

  \pnote{
    Modèle déductif plus efficaces sur les protocoles de routage.
  }
  \pnote{
    - Mentionner la différence entre un système mono-utilisation (thermostat) vs. un sytème globalement intégré (Amazon Echo / SCADA).
  }
  \pnote{
    - Les protocoles visent différents usages et contraintes il est plus rentable d'avoir des systèmes généralistes que outils en silo.
  }
  \pnote{
    - On fournit autant de services généralistes que l'on peut avant d'aller vers le particulier.
  }
\end{frame}

\begin{frame}{Autres contributions}
  
  \begin{block}{Interconnexions de passerelles en pair à pair pour la découverte de services}
    \begin{itemize}
      \item \textit{{Cirani, Davoli, Ferrari, L{\'e}one, Medagliani, Picone, Veltri} A scalable and self-configuring architecture for service discovery in
        the internet of things. (IEEE Internet of Things Journal)}

    \end{itemize}
  \end{block}

  \begin{block}{Contributions open source}
    \begin{itemize}
      \item Mise en place de l'intégration continue au sein du projet Contiki (\href{https://contiki-os.blogspot.fr/2012/12/contiki-regression-tests-9-hardware.html}{link})
      \item Documentation et automatisation d'expériences reproductibles sur les LLNs (Makesense) (\href{https://github.com/sieben/makesense}{link})
      \item \textit{L{\'e}one, Leguay, Medagliani, Chaudet. Makesense: Automating WSN experiments and simulations.
              EWSN, 2015.}
    \end{itemize}
  \end{block}

% \begin{itemize}

%   \item Fadwa Boubekeur, L{\'e}lia Blin, Remy Leone, and Paolo Medagliani. Bounding degrees on rpl.
%   In {\em Proceedings of the 11th ACM Symposium on QoS and Security for
%   Wireless and Mobile Networks}, pages 123--130. ACM, 2015.

% \end{itemize}

\end{frame}

\frame[containsverbatim]{
  \frametitle{Remarques? Questions?}

  Merci pour votre attention.
  \begin{itemize}
    \item remy.leone@telecom-paristech.fr
  \end{itemize}
  \pnote{
    - Remerciements
  }
}

