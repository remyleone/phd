\section{Conclusion}

\begin{frame}\frametitle{Conclusion}

  \begin{block}{À retenir}
    \begin{itemize}
      \item Compromis consommation / QoS
      % \item Réguler requetes LLN $\implies$ économies d'énergie
      \item Originalité: attaquer le problème au niveau de la passerelle
        \end{itemize}    
  \end{block}
  

  \begin{block}{Supervision peu intrusive}
      \begin{itemize}
        % \item La connaissance de la topologie améliore la précision
        % \item Calibration par strobing validée mais insuffisante
        % \item Mesures actives nécessaires pour cette configuration
        \item Connaitre le routage améliore l'estimation
        \item Mesure passive pour économiser de l'énergie
        \item Mesure active pour recalibrer
      \end{itemize}
  \end{block}

  \begin{block}{Cache}

    \begin{itemize}

      %[Pourquoi]
      \item Impact du cache sur la durée de vie
      \item Compromis réglable entre fraicheur et consommation

    \end{itemize}

  \end{block}


\end{frame}

\begin{frame}{Ouvertures}
  
  \begin{block}{Mesure implicite}
    \begin{itemize}
      \item Autres topologies
      \item Modélisation plus fine des protocoles de routage (RPL)
      \item Apprentissage du biais et adaptation de la fréquence des mesures actives
    \end{itemize}
  \end{block}


  \begin{block}{Cache}
    \begin{itemize}
      % \item Autres résolutions du problème multi-objectifs 
      \item Optimisations supplémentaires: latence, popularité d'une requete
      \item Autres mesures de satisfaction
      % \item Prise en compte du délai de réponse
    \end{itemize}
  \end{block}

\begin{block}{Autres améliorations}
    \begin{itemize}
        \item Adapter le modèle aux couches MAC synchrones (ex: TSCH)
    \end{itemize}
  \end{block}

  % \begin{block}{Approche dynamique}
  %   \begin{itemize}
  %     \item Reconfiguration dynamique
  %   \end{itemize}
  % \end{block}

  % \begin{alertblock}{Fonctionnalités supplémentaire des passerelles}
  %   \begin{itemize}
  %     \item Interopérabilité logicielle
  %     \item Ajout de services
  %     \item Aggrégation et découverte de services
  %   \end{itemize}
  % \end{alertblock}

  \pnote{
    Modèle déductif plus efficaces sur les protocoles de routage.
  }
  \pnote{
    - Mentionner la différence entre un système mono-utilisation (thermostat) vs. un sytème globalement intégré (Amazon Echo / SCADA).
  }
  \pnote{
    - Les protocoles visent différents usages et contraintes il est plus rentable d'avoir des systèmes généralistes que outils en silo.
  }
  \pnote{
    - On fournit autant de services généralistes que l'on peut avant d'aller vers le particulier.
  }
\end{frame}
%
\begin{frame}[allowframebreaks]{Publications}
  % \begin{block}{Supervision}
    % \begin{itemize}
  %     \item \textit{Tee: Traffic-based energy estimators for duty-cycled Wireless Sensor Networks} Léone, Leguay, Medagliani, Chaudet. IEEE International Conference on Communication (ICC)
  % %   \end{itemize}
  % % \end{block}
  % % \begin{block}{Cache adaptatif}
  % %   \begin{itemize}
  %     \item \textit{Optimizing QoS in Wireless Sensors Networks using a Caching Platform}, Leone, Medagliani, Leguay. Sensornets 2013
  %     \item \textit{Optimisation de la qualité de service par l'utilisation de mémoire cache}, Leone, Medagliani, Leguay. Algotel 2013

  %     \item \textit{A scalable and self-configuring architecture for service discovery in the internet of things} Cirani, Davoli, Ferrari, L{\'e}one, Medagliani, Picone, Veltri (IEEE Internet of Things Journal)

% \tiny{
% \item R{\'e}my Leone, Paolo Medagliani, and J{\'e}r{\'e}mie Leguay.
% \newblock Optimizing qos in wireless sensors networks using a caching platform.
% \newblock In {\em Sensornets 2013}, page~56, 2013.
  
% }

  \bibliography{mine}
  \bibliographystyle{ieeetr}

    \nocite{*}

    % \end{itemize}
  % \end{block}
\end{frame}

\begin{frame}{Autres contributions}
  
  % \begin{block}{Interconnexions de passerelles en pair à pair pour la découverte de services}
  %   \begin{itemize}
  %   \end{itemize}
  % \end{block}

  \begin{block}{Contributions open source}
    \begin{itemize}
      \item Proxy inverse disponible par le projet Calipso (Smart Parking) % (\href{https://github.com/sics-iot/calipso}{link})
      \item Mise en place de l'intégration continue au sein du projet Contiki %(\href{https://contiki-os.blogspot.fr/2012/12/contiki-regression-tests-9-hardware.html}{link})
      \item Documentation et automatisation d'expériences reproductibles sur les LLNs (Makesense) % (\href{https://github.com/sieben/makesense}{link})
      % \item \textit{L{\'e}one, Leguay, Medagliani, Chaudet. Makesense: Automating WSN experiments and simulations.
      %         EWSN, 2015.}
    \end{itemize}
  \end{block}

% \begin{itemize}

%   \item Fadwa Boubekeur, L{\'e}lia Blin, Remy Leone, and Paolo Medagliani. Bounding degrees on rpl.
%   In {\em Proceedings of the 11th ACM Symposium on QoS and Security for
%   Wireless and Mobile Networks}, pages 123--130. ACM, 2015.

% \end{itemize}

\end{frame}

\frame[containsverbatim]{
  \frametitle{Remarques? Questions?}

  Merci pour votre attention.
  \begin{itemize}
    \item remy.leone@telecom-paristech.fr
  \end{itemize}
  \pnote{
    - Remerciements
  }
  \label{slide:end}
}
