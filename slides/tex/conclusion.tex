\section{Conclusion}

\begin{frame}\frametitle{Conclusion}

  \begin{alertblock}{Contributions}
    \begin{itemize}
      \item Les passerelles peuvent héberger des fonctionnalités réseau avancées
      \item La supervision passive peut fournir des estimations de consommation énergétique de manière transparente
      \item Un cache adaptatif peut adapter la quantité de requêtes entrantes dans un réseau contraint
      % \item La recherche reproductible et des cycles de développement courts sont possibles dans ce champ d'étude
    \end{itemize}
  \end{alertblock}

\end{frame}

\begin{frame}\frametitle{Ouvertures}
  \begin{block}{Raison du succès des passerelles}
    \begin{itemize}
      \item Différents compromis (bande passante, consommation, portée)
      \item Plateformes hétérogènes
    \end{itemize}
  \end{block}

  \begin{alertblock}{Fonctionnalités supplémentaire des passerelles}
    \begin{itemize}
      \item Interopérabilité logicielle
      \item Ajout de services
      \item Aggrégation et découverte de services
    \end{itemize}
  \end{alertblock}


  \pnote{
    - Mentionner la différence entre un système mono-utilisation (thermostat) vs. un sytème globalement intégré (Amazon Echo / SCADA).
  }
  \pnote{
    - Les protocoles visent différents usages et contraintes il est plus rentable d'avoir des systèmes généralistes que outils en silo.
  }
  \pnote{
    - On fournit autant de services généralistes que l'on peut avant d'aller vers le particulier.
  }
  \pnote{---}
  \pnote{
  - L'utilisation de testbed permet de se lancer facilement et développer de nouveaux services
  }
  \pnote{- L'intégration est difficile car de nombreuses pannes peuvent arriver}
   \pnote{ 
   - Une couverture de test systématique peut aider à mitiger ces risques
   }
   \pnote{
    - Démonstration de l'automatisation.
  }
\end{frame}



\begin{frame}{Autres travaux}
  
  \begin{block}{Découverte de service autoconfiguré}
    \begin{itemize}
      \item Interconnexions de passerelles en pair à pair pour la découverte de services
    \end{itemize}
  %   Simone Cirani, Luca Davoli, Gianluigi Ferrari, R{\'e}my L{\'e}one, Paolo
  % Medagliani, Marco Picone, and Luca Veltri.
  % A scalable and self-configuring architecture for service discovery in
  % the internet of things, 2014.
  \end{block}

  \begin{alertblock}{Recherche reproductible dans les réseaux de capteurs}
    \begin{itemize}
      \item Documentation et automatisation d'expériences sur les LLNs
    \end{itemize}
%     R{\'e}my L{\'e}one, J{\'e}r{\'e}mie Leguay, Paolo Medagliani, and Claude
%   Chaudet.
% \newblock Demo Abstract: Automating WSN experiments and simulations.
% \newblock In {\em EWSN}, 2015.
  \end{alertblock}

% \begin{itemize}

%   \item Fadwa Boubekeur, L{\'e}lia Blin, Remy Leone, and Paolo Medagliani. Bounding degrees on rpl.
%   In {\em Proceedings of the 11th ACM Symposium on QoS and Security for
%   Wireless and Mobile Networks}, pages 123--130. ACM, 2015.

% \end{itemize}

\end{frame}

\frame[containsverbatim]{
  \frametitle{Remarques? Questions?}

  Merci pour votre attention.
  \begin{itemize}
    \item remy.leone@telecom-paristech.fr
  \end{itemize}
  \pnote{
    - Remerciements
  }
}

