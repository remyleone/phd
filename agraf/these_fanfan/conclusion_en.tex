% !TeX spellcheck = en_US
\chapterstar{Conclusion and perspectives}\label{sec:conclusion_these}
\markboth{Conclusion and perspectives}{Conclusion and perspectives}

At the time of concluding this memoir, we would like to reconsider various concepts encountered, with particular attention to perspectives they suggest for future work.

\medskip
In the introduction to this memoir, we recalled what we think are the founding paradoxes of voting theory. To the relatively traditional triptych consisting of Condorcet's paradox \citep{condorcet1785essai}, Arrow's theorem \citep{arrow1950difficulty} and Gibbard-Satterthwaite theorem \citep{gibbard1973manipulation, satterthwaite1975strategyproofness}, we believe that we must add the impossibility results of interpersonal comparison of utilities, which limit one of the few possible escapes from Arrow's theorem and, thus, reinforce the negative outcome it states. We recalled that the results by \cite{gibbard1973manipulation} does not apply only to ordinal voting systems, in contrast to a weaker formulation of the theorem which is sometimes presented. We also discussed Gibbard's nondeterministic theorems \citep{gibbard1977chance, gibbard1978lotteries}, which fully characterizes the non-manipulability for voting systems using the chance; but we believe that the resulting systems (especially \emph{random dictatorship}) must be limited to applications where elections are frequent and where the result has a moderate impact. In contrast, our work concerns deterministic voting systems.

We insisted, as does \cite{gibbard1973manipulation}, on the fact that the deep problem related to manipulability is the defect of \emph{straightforwardness}, that is to say the fact that voters can not always optimally defend their opinion without knowing the ballots of other voters. This causes many doubts when voting, asymmetry \emph{de facto} between informed voters and those who are not, problems of legitimacy for the outcome and a questionable power granted to sources of information such as pollsters. But we also recalled that a low manipulability can be seen as an approached straightforwardness, and studying the manipulability may therefore be addressed as a technical means to tackle straightforwardness.

While respecting the viewpoint that condemns the action of \emph{manipulation}, we proposed a vision perhaps less orthodox of defending the strategic attitude of voters and to condemn \emph{manipulability}, not because you might get a manipulated result, but because of lack of information or voter strategic approach, we risk getting a sincere result which is not a strong Nash equilibrium (SNE), which would imply regret for some voters and would question the legitimacy of the result. We have argued that an easy manipulation is more desirable than a difficult one; but the easiest ``manipulation'' occurs when sincere voting is strategically optimal, that is to say, when the configuration is not manipulable.

\medskip
In chapter~\ref{sec:formalisme}, we defined our formalism, especially the notion of \emph{electoral space}, which allows the study in full generality voting systems based on orders but also approval values, grades or any other type of information. We have given some examples of electoral spaces to show the richness of such a framework but this list is far from exhaustive. We defined \emph{general voting systems}, inspired by the framework of \emph{game forms} by \cite{gibbard1973manipulation}. We focused on \emph{state-based voting systems}, which allow to overcome the issue of a problematic definition of a canonical sincere voting. Indeed, we have shown that, with a view to limit the manipulability, we can restrict our investigation to SBVS, thus generalizing a result by \cite{moulin1978strategie} that applied to ordinal electoral spaces only. We showed how we could establish correspondences between the general systems and SBVS by the state-based version of a general voting system and the canonical implementation of a SBVS.

We took advantage of section~\ref{sec:anonymat} to characterize the values of the pair $(V, C)$ that allow for anonymous and neutral voting systems in the electoral space of strict total orders (proposition~\ref{thm:neutr_anon_cns}), which is a variation of a result by \cite{moulin1978strategie} which dealt with voting systems that are, in addition, \emph{efficient}.

In section~\ref{sec:condorcet_definitions}, we recalled the usual notions of Condorcet winner and weak Condorcet winner, and we defined the concepts of absolute Condorcet winner and Condorcet-admissible candidate, whose the rest of this memoir showed the relevance for the study of manipulability.

In section~\ref{sec:zoologie}, we presented the voting systems studied in this memoir. In particular, we defined the \emph{Iterated Bucklin's method} (IB) and \emph{Condorcet method with sum of defeats} (CSD), which is inspired by Dodgson's method. The system \emph{IRV Duels} (IRVD), a variant of the \emph {Instant-Runoff Voting} that is a contribution from Laurent Viennot. We have also defined the system \emph{Condorcet-Dean} and \emph{Condorcet-dictatorship} which, by their simplicity, provide good examples and counterexamples for various theoretical results.

\medskip
Among the different concepts introduced in chapter~\ref{sec:formalisme}, we would like to discuss one last time two of them in particular.

Firstly, through anonymity (section~\ref{sec:anonymat}), and later through the majority unison ballot criterion, \bulmajuni{} (section~\ref{sec:criteres_majoritaires_definition}), we saw that one could define \emph{group actions of the permutations over an electoral space} and \emph{isomorphisms of voting systems}. In general, it seems interesting to define concepts that are stable by isomorphism in order to have properties that are intrinsic, and not related to the labeling of objects. Several applications may be obtained by such an approach.
\begin{enumerate}
\item For example, we believe that for some voting systems, but not all, it is possible to define a canonical sincere ballot as a possible strategy that is optimal in all neutral culture, that is to say, stable by a certain class of isomorphisms. After clearing the theoretical difficulties to rigorously define the concept in any electoral space, it would be interesting to study what voting systems have such canonical sincere ballot and for which of them this is not the case.
\item Furthermore, the notion of isomorphism can be extended to a more general notion of morphism of voting systems that is not necessarily an isomorphism, that is to say, not necessarily bijective. This opens up opportunities to examine transformations of voting systems or to establish correspondences between some known systems, which could explain some similar behaviors.
\end{enumerate}

Secondly, in the context of this memoir, although we use the word \emph{preference} for conciseness, the exact meaning of the binary relation $c \Plusque{} \winner$ is: if candidate~$\winner$ is the sincere winner, then voter~$ v $ is inclined to act so that candidate $ c $ is the winner instead. By allowing preferences that are not antisymmetric but arbitrary binary relations, we were able to model a wide variety of behaviors of voters.
We mainly seen examples of preferences that violate transitivity, or even negative-transitivity (examples ~\ref{ex:utilites_epsilon}, \ref{ex:interval_graph}, \ref{ex:preferences_multicriteres}). Another possible advantage of our framework is to combine manipulability in the usual sense (where voters manipulate for the result to be more in line with their personal opinions) and corruption (where some voters are apt to manipulate in one direction or the other, as they receive or not a bribe). We have discussed this possibility quickly in section~\ref{sec:manipulabilite_definition} but there is a much broader field of study in this direction. One can imagine, for example, that for each voter and each pair of candidates $ (\winner, c) $, there is a minimum amount of money for which the voter agrees to participate in a manipulation for~$ c $ against~$\winner$.
Thus, the matrix of her binary relation is replaced by a matrix of non-negative real costs, in which $ +\infty $ corresponds to \textbf{False} (not interested) and 0 is \textbf{True} (always interested, even without financial incentive).
We can then examine with what total budget it is possible to make a given candidate win, as is customary in bribery problems.

\medskip
In chapter~\ref{sec:condorcification}, we have defined the \emph{informed majority coalition criterion} (\cminf{}), which is the weakest of the majority criteria we defined thereafter. We proved the Condorcification theorems, weak \ref{thm_Condorcifie_aussi_bon} and strong~\ref{thm:cond_strictement_meilleur}: if a voting system meets \cminf{}, then its Condorcification is at most as manipulable. If, furthermore, it does not meet the \emph{resistant-Condorcet criterion} (\rcond{}), then its Condorcification is strictly less manipulable. As we have seen, the strong version of the theorem applies to most of the usual voting systems.

To demonstrate the strong Condorcification theorem~\ref{thm:cond_strictement_meilleur}, we introduced and characterized the notion of resistant Condorcet winner. This concept provides an upper limit of manipulability (in sense of inclusion) for Condorcet systems, which we have shown to be tight for $ C \geq  6$; but we later saw that it was not tight for $ C = 3 $ (chapter~\ref{sec:mds_optimaux}). It would be interesting to study the intermediary values of~$ C $ in order to also have a tight upper bound in this case.

We showed that in the general case, it is the \emph{absolute} Condorcification that makes Condorcification theorems work. However, we also saw that a \emph{relative} Condorcification may, for Plurality or the two-round system, reduce manipulability, even compared to the absolute Condorcification. However, the relative Condorcification does not reduce manipulability for IRV, the majority judgment, approval voting or range voting. It would be useful to extend this investigation to all classic voting systems and, if possible, to identify a condition (that applies to as many systems as possible) that is sufficient so that the relative Condorcification is at most as manipulable as the initial voting system, or even at most as manipulable as the absolute Condorcification.

As we saw later in the simulations (chapters \ref{sec:simulations_spheroidal} to~\ref{sec:mds_optimaux}), the manipulability reduction offered by Condorcification is not necessarily quantitatively very important, especially when comparing IRV and its Condorcification CIRV. It seems that the main consequences of these Condorcification theorems are to be found in the corollaries \ref{thm:minimum_set} and~\ref{thm:minimum_manip_cond}: to research a voting system of minimal manipulability within class \cminf{}, we can restrict the study to Condorcet systems.

\medskip
In chapter~\ref{sec:criteres_majoritaires}, we have established a hierarchy of various majoritarian criteria for a voting system and we showed their links with the existence and some form of uniqueness of strong Nash equilibria. In particular, we extended the results by \cite{sertel2004strong} and \cite{brill2015strategic}, showing that the criterion \cminf{} is equivalent to the criterion of \emph{restriction of possible SNE to Condorcet-admissible candidates} \renfadm{}, according to which any winner of a SNE is a Condorcet-admissible candidate (``uniqueness'' of the equilibrium). We also have studied the criterion of \emph{existence of an SNE for any Condorcet winner} \eenfcond{}, according to which there is an SNE in any Condorcet configuration of preferences (existence of an equilibrium). We have shown that \eenfcond{} is included in the class \cmign{} of systems meeting the \emph{ignorant majority coalition criterion} and contains the class \bulmaj{} of systems meeting the \emph{majority ballot criterion} (within reasonable electoral space) and that these inclusions are strict in general. It would be interesting, in the future, to have a simple characterization of the voting systems that verify \eenfcond{}.

By the way, we showed that the criterion of \emph{restriction of possible SNE to Condorcet winners} \renfcond{} and the criterion of \emph{existence of an SNE for any Condorcet-admissible candidate} \eenfadm{} are less ``natural'' than their weaker versions, \renfadm{} and \eenfcond{} respectively, because they are rarely met by classic voting systems, because they do not have simple relations of inclusion with the other criteria and because these two criteria are generally incompatible with each other.

We examined the criteria verified by the usual voting systems, compiling classic literature results and presenting original results, particularly, on one hand, about the criteria \cminf{} and \rcond{} and, on the other hand, about all criteria for Iterated Bucklin's method. Despite our efforts, some questions remain open: to find a necessary and sufficient simple condition under which a positional scoring rule (PSR) meets \cminf{} and the conditions under which an iterated PSR (IPSR-ES, with simple elimination, or IPSR-EA, with elimination based on the average) meets \rcond{}.

In section~\ref{sec:informationnel}, we informally commented links between majority criteria and the concept of information exchange, particularly to achieve an equilibrium or, at least, to find the same winning candidate as in an equilibrium. There, again, a whole field of investigation as possible, in connection with the field of distributed algorithms. For example, one can ask this question: given a voting system, does there exist an algorithm that achieves an SNE when it exists? In synchronous execution without fault, this is obviously possible, since we can emulate a non-distributed behavior; but what about asynchronous execution? Is it possible to establish an algorithm in a way that is resistant to fault, that is to say, in this context, to the non-cooperation of some voters? And above all, what is the best complexity in exchange of information that such an algorithm may have? In worst case? On average?

\medskip
In chapter~\ref{sec:condorcification_generalisee}, we defined \emph{generalized Condorcification}, using on the notion of \emph{family}, which is inspired by the theory of simple games. This enabled us, in particular, to apply a transformation inspired by Condorcification to voting systems that do not meet some majority criteria in particular (generalized Condorcification theorem~\ref{thm_condorcification_generalisee}). We compared the manipulability of Condorcifications performed with two different families (compared Condorcification theorem~\ref{thm:condorcification_comparee}). Finally, we defined the \emph{maximal family} of a voting system, that is to say, the family of coalitions that can manipulate (in an informed way) for this or that candidate. We have shown that under certain assumptions, Condorcification using this maximal family is the least manipulable of the Condorcifications conducted with a family $ \ens {M} $ such that the system meets $ \mcinf{} $ (maximal Condorcification theorem~\ref{thm:condorcification_maximale}). We were able to establish that for classic systems that meet \cminf{}, the majoritarian Condorcification is the maximal Condorcification, and therefore it is optimal in some sense. In particular, we showed that (for an odd number of voters and antisymmetric preferences), Condorcet systems are their own maximal Condorcification. We also provided examples of generalized Condorcification for various voting systems violating non marginally anonymity and/or neutrality, which have shown the wide application of the theorems of this chapter.

To extend these results, it would be ideal to find a theorem of the following form: the maximal Condorcification of $ f $ is the least manipulable voting system (in the large sense) among those who share a certain well-chosen property with $ f $. The maximal Condorcification theorem we have is, in a way, of this form, but the common property is to be a generalized Condorcification of $ f $ by a family $ \ens{M} $ such that $ f $ meets \cminf{}. It would be interesting to have a simpler and more intuitive property.

By the way, families that we consider are indexed by a candidate~$ c $ and describe the coalitions that can make~$ c $ win (usually in an informed way). This approach has the advantage of allowing a direct translation of the reasoning made for the weak Condorcification theorem~\ref{thm_Condorcifie_aussi_bon} to generalized Condorcification theorem~\ref{thm_condorcification_generalisee}. But one could also consider generalizing this \emph{vector} of coalitions collections indexed by~$ c $, to a \emph{matrix} of coalitions collections indexed by the winner to dethrone~$\winner $ and the candidate for whom you wish to manipulate~$ c $.

\medskip
In chapter~\ref{sec:slicing}, we showed the slicing theorem~\ref{thm_slicing}: if the culture is \emph{decomposable}, then for any voting system, there is one of its slices that is at most as manipulable as the original system (in the probabilistic sense). In particular, we have shown that this is the case when voters are independent by proving a more general result of probability, the lemma of the complementary random variable~\ref{thm:donnee_alea_sup}. We examined the possible generalizations of this theorem and showed that we can not require a manipulability reduction in the sense of inclusion and that we can not purely and simply remove the assumption of decomposability. The main open question that we have left is whether you can require a weaker assumption that decomposability. In particular, it would be interesting to know whether condition~\eqref{eq:cn_decompos_ES} from proposition~\ref{thm:CN_decomposabilite_ES}, which is necessary but not sufficient for decomposability, is still sufficient to lead to the same conclusions as in the slicing theorem. The ideal would be to have a tight version of the theorem, that is to say, to find a condition on the culture, not only sufficient, but also necessary so that any SBVS admits a slice that is at most as manipulable as the original system.

Slicing theorem~\ref{thm_slicing} is not constructive, which does not lead to an immediate practical application. But, in our opinion, its main consequence is the optimality theorem~\ref{thm:general_optimality}: if the culture is decomposable, then a system that is optimal (in the probabilistic sense) among ordinal Condorcet voting systems is also optimal in the much broader class of systems that meet \cminf{} and can fail to be ordinal. If one's objective is to reduce the manipulability, this means that systems like range voting, the majority judgment or approval voting are \emph{a priori} unpromising, before even considering their results in simulations. That said, if one's objective is a certain algorithmic simplicity in the identification of strategic voting, approval voting has interesting properties, which we have already mentioned.

\medskip
In chapter~\ref{sec:svvamp}, we presented \svvamp{}, \emph{Simulator of Various Voting Algorithms in Manipulating Populations}, a Python package dedicated to the study of voting systems and especially their manipulability. The populations of voters can be characterized by strict weak orders or utilities; you can import them from external files or generate them by a variety of random models. This simulator implements coalition manipulation (CM) and various variants (informed coalition manipulation ICM, unison manipulation UM, trivial manipulation TM), and individual manipulation (IM), independence of irrelevant alternatives (IIA) and the Condorcet concepts presented in this memoir.

Generic methods make it possible to implement new voting systems; and dedicated algorithms are implemented for some of them. We gathered algorithms corresponding to the state of the art and have developed original methods, especially for IRV and its variants, given the particular interest of these voting systems to achieve low manipulability.

Such software is inherently a work in constant evolution, where developments are always possible. For populations, it would be possible to implement more general models, especially non-transitive preferences. It would also be interesting to allow to vary the tie-breaking rule used for each voting system, but one should be aware that this option entails significant difficulties for manipulation algorithms. Finally, the main source of improvement would be to implement dedicated manipulation algorithms for voting systems that do not already have some and currently use generic algorithms. Regarding IRV, the main improvement would be to find a polynomial heuristic that allows to certify the non-manipulability in a high proportion of non-manipulatable configurations, in the same way that our heuristic identifies a significant proportion of manipulable configurations .

Given the good performance of IRV in terms of manipulation, it would be interesting to implement various variants of IRV in \svvamp{} to compare their manipulability. We have already studied IRVD and of course CIRV. Currently, two other variants IRV are implemented in \svvamp{}: IRVA, which is the IPSR-EA associated with Plurality, and another system called \emph{Instant-Condorcet Runoff Voting} (ICRV). We preferred not to include these systems in our study for the following reason: as these systems usually have a very low TM rate (like IRV), the generic algorithm for CM usually gives a pretty high algorithmic uncertainty, which does not make it possible compare the manipulability of these systems to the others. In the future, it would be interesting to implement dedicated algorithms for these systems and add other variants of IRV, such as those mentioned by \cite{green2011four} (which includes ICRV).

One goal of \svvamp{}, in the long term, is to measure manipulability precisely, not only for a culture but for a given profile. For this, we have developed a methodology by adding random noise, but it would be interesting to develop other metrics. For example, a family of manipulability indicators proposed in the literature is based on the size of the coalitions (which is already partially implemented in \svvamp{}). Another approach seems interesting, inspired by problems of corruption and reflections on the analysis of real experiences: the notion of \emph{manipulation threshold}. For each candidate $ c $ different from the winner $\winner$ and for each real number $\varepsilon$, we can raise the following question: considering voters where $ c $ brings higher utility improvement than $ \varepsilon $ compared to $ \winner $, is this coalition able to make~$ c $ win? By authorizing negative values of $ \varepsilon $, the coalition can contain all voters, so there is a threshold $ \varepsilon $ below which it is actually possible\footnote{Here, we use the implicit assumption that the voting system considered is surjective, that is to say, that any candidate is actually eligible. This condition is met by all reasonable voting systems (except possibly Veto if $ C> V + $ 1).}. This threshold contains not only information of manipulability for~$ c $ (which is true \ssi the threshold is positive) but it also measures the ease of manipulation or a form of distance to a manipulable configuration.

Note that \svvamp{} is designed to perform a wide range of simulations, only part of which has been exploited herein. The most important opportunities, which we have ignored in order to devote ourselves to manipulation by coalition, concern the individual manipulation. Indeed, dedicated algorithms are implemented for almost as many voting systems as in the case of manipulation by coalition: this is the case for the majority judgment, Maximin, Plurality, the two-round system (TR), Veto, approval voting, range voting, exhaustive ballot (EB), IRV and Borda, Bucklin, Coombs and Schulze's methods. It would be interesting to use \svvamp{} to study individual manipulation.

\medskip
In chapters \ref{sec:simulations_spheroidal} and~\ref{sec:simulations_unidim}, we used \svvamp{} to study the manipulability of various voting systems in spheroidal cultures, using for the first time the Von Mises-Fisher model, and then in cultures based on a political spectrum. This allowed us to verify some results known in the literature and offer a number of conjectures about the monotony and the limit of manipulability rates based on certain parameters, particularly $ C $ and $ V $. While we have shown some of them, others remain conjectures.

Among these conjectures, the one that seems the most accessible is that the manipulability rate of reasonable voting systems (\cminf{}) tends to~1 when $ C \to \infty $ in spherical culture (conjecture~\ref{conj:sph_MC_limite_C_infini}). We have demonstrated this result for an odd number of voters. To extend the result to $ V $ even, it would be sufficient to prove that in impartial culture, the probability of having a Condorcet-admissible candidate (that is to say, a weak Condorcet winner) tends to 0 when $ C \to \infty $. Moreover, we proved that the CM rate of Veto (with the tie-breaking rule used in \svvamp{}) does not tend to~1 when $ C \to +\infty $ in impartial culture, even if the simulations suggest a limit close to~1.

It also seems interesting to examine the limits of the manipulability rates in spherical culture when the number of voters is very large (conjectures \ref{conj:V_infini_MC_tend_vers_1} and \ref{conj:V_infini_MC_tend_pas_vers_1}). Although the spherical culture is not intended to be a descriptive model, it offers a normative baseline that can be seen as a worst case, since it is the most disorderly (entropy is maximum). Thus, it is interesting to know what voting systems do not have a manipulability rate tending to~1 when $ V \ to \infty $ in this culture which is rather unfavorable \emph{a priori}\footnote{About this idea, one may in particular consult \cite{tsetlin2003impartialculture}.}. We conjectured that among the studied voting systems, only IRV, EB, CIRV and Veto have a CM rate that tends to a limit that is different from~1 when $ V \to +\infty $ in the impartial culture.

We observed oscillatory phenomena for CM rates, which are more pronounced in cultures of unidimensional political spectrum than in spheroidal cultures, and we have proposed a qualitative explanation of this. For the majority judgment and Bucklin's method, we hypothesized that this behavior is amplified by the fact of considering an unfavorable median, which tends to make them more manipulable when the number of voters is even. For future work, it would be interesting to examine variations of such voting systems where favorable median is used, to see if it reduces the manipulability in this case.

When the population tends to become uniform (sections \ref{sec:vmf_concentration}, \ref{sec:vmf_position_pole}, \ref{sec:vmf_positions_des_poles} and~\ref{sec:simus_gauss_decalage}), we found, not surprisingly , that most voting systems become less manipulable. However, we saw that some systems are much less reactive in terms of manipulability reduction in this case, especially Veto, approval voting, range voting, Borda's method and the majority judgment. Regarding Veto (with the tie-breaking rule used in \svvamp{}), we have even shown that its manipulability does not tend to~0 when voters have identical preferences, unlike all unanimous voting systems.

By the way, we confirmed the importance of the resistant Condorcet winner for the non-manipulability of Condorcet voting systems, especially through the upper bound given in section~\ref{sec:borne_sup_manip_pour_les_cond}. For this reason, it would be interesting to prove the decrease of its probability of existence relatively to~$ C $ in spherical culture. In general, it would be interesting to study the probability of existence of a resistant Condorcet winner in various contexts, as \cite{gehrlein2006condorcet} and others did for a Condorcet winner or a weak Condorcet winner.

Finally, these simulations gave a good overview of the comparative performance of different voting systems. In the spheroidal cultures and multidimensional political spectrum models, CIRV, IRV and~EB show the best performances. In unidimensional political spectrum cultures, Schulze's method, Maximin, the two-round system, Bucklin's method and IRVD show promising results that encourage further study of these systems in the future. Generally, range voting, approval voting and Borda's method are very manipulable; in particular, they are often more manipulable than all Condorcet systems, as can be seen by comparing with the rate of existence of a resistant Condorcet winner.

\medskip
In chapter~\ref{sec:simulations_expe}, we analyzed the results of 168~actual experiments, including 17 from original experiments and others from PrefLib database. Among these data sets, it is particularly interesting to consider:
\begin{itemize}
\item those based on cardinal data because they allow to deepen the comparison between ordinal and cardinal systems and they also help to naturally consider issues based on the notion of utility, like the manipulability threshold described above;
\item those that do not relate to an election scenario and whose strategic importance is \emph{a priori} low (like bdtheque experiences) because these data sets can be expected to have opinions as sincere as possible.
\end{itemize}
For future work, it would be interesting to expand this corpus as much as possible.
  
In all experiments analyzed, CIRV, IRV and EB are distinguished by CM rates that are always lower than that of other voting systems. In TM, the same result holds for CIRV and almost always for IRV and~EB. But we have seen in chapter~\ref{sec:simulations_unidim} that the performance of these voting systems are less good (compared to other voting systems) in cultures based on a unidimensional political spectrum than in other cultures.
So, we see some opposition between the performance of IRV-type systems in our real data sets and unidimensional spectra. It would be interesting to carry out more experiments on large-scale political elections to test the resistance of CIRV, IRV and EB to manipulation in these application cases that are \emph{a priori} rather unfavorable and check how the assumption of unidimensional spectrum is credible, especially in terms of manipulability.

By the way, we confirmed experimentally that it is very common to have a Condorcet winner, joining, in this, various previous results. In practice, this limits the severity of the Condorcet paradox and Arrow's theorem. Indeed, there is a canonical way to synthesize voter preferences anonymously, neutrally and monotonically with having a lot of good properties: it is the victory relation in the matrix of duels. The only flaw of this relation is that it is not necessarily transitive. But the experiments show that it often has a maximal element (the Condorcet winner), making it a natural winner, with good properties such as IIA. For this reason, we advance the idea that the behavior of a voting system in non-Condorcet configurations should not be considered in terms of the relevance of the elected candidate when such a configuration occurs by sincere voting (because in this case, there is no perfect solution), but by the impact it has on the opportunities for manipulation of Condorcet configurations.

\medskip
This is precisely what we did in chapter~\ref{sec:mds_optimaux}, where we introduced the \emph{opportunity graph} of an electoral space and we used it to study optimal voting systems, that is to say, whose manipulability rate is minimal in a certain class. We considered \cminf{}, so we could restrict ourselves to systems meeting \cond{} through Condorcification theorems, but the reasoning we used is valid in all generality.

This object raises several theoretical questions. On the set of all labeled and weighted multigraphs as the typical graph of figure~\ref{fig:graphe_pointage_abstrait}, is it true that solving the problem of minimal contamination is \complexNP-hard? If we restrict to the opportunity graphs obtained for a certain class of electoral spaces, such as those of strict total orders for all values of~$ C $ and odd values of~$ V $, is is true that the problem remains \complexNP-hard? Can we develop a polynomial algorithm that allows a better approximation than the greedy algorithm we presented?

In the case of $C = 3$~candidates and $V = 3, 5$ or 7~voters, we found that CIRV, provided it are equipped with an adequate tie-breaking rule, is always optimal. It is therefore natural to ask if it is the same for any number of voters and $C = 3$. For $C = 4$ candidates, we already know that this is not generally the case, since we established that this is not true for $V = 3$ and $C = 4$.

It would be interesting to extend the search for optimal systems to other parameter values. In our view, the main purpose of this approach is not necessarily to use these systems in practice: one can rarely guarantee in advance that there will be a number of voters and a number of candidates for which we are able to solve the problem. The goal is rather to use the few cases where the optimal systems can be exhibited to better observe how the manipulability behaves and understand what can makes a system little manipulable.

\medskip
In summary, we have introduced a unified formalism to study ordinal voting systems and non-ordinal ones. We proposed different tools to transform a voting system in order to obtain a lower manipulability: the reduction of a voting system to its state-based version, Condorcification, relative Condorcification in some cases, generalized Condorcification and slicing. These tools helped in particular to establish optimality theorems that suggest further research on ordinal Condorcet systems. Furthermore, we have established various criteria for a voting system and showed the deep connections between these criteria, the manipulability, Nash equilibria and the concept of information exchange. We have laid the first stones of the building of optimal Condorcet voting systems, allowing to have a yardstick of manipulability for moderate values of the parameters and provide a better insight of the reasons that make a system little manipulable. In this, we have made some steps on the path leading us ``towards less manipulable voting systems''.

With our software package \svvamp{}, we have also been able to study the manipulability from a quantitative point of view and propose some answers to the problems posed in the introduction. In particular: yes, manipulability is a common phenomenon in artificial cultures and in the real world, it is not just a theoretical possibility. It raises problems which are much more frequent than the Condorcet paradox. We have also been able to measure the very different vulnerabilities of the various voting systems to this phenomenon, and we have showed the particularly dramatic supremacy of CIRV, IRV and EB in all studied real-life experiments, which encourages to keep on studying other voting systems from this family.

\medskip
All these results have not necessarily the ambition of an immediate use in large-scale elections, although in the case of IRV, we have a system that is already used in various countries for political elections. The progress of electronic voting, including about security issues, will soon make is possible to consider using voting systems with more complex counting than is possible with the manual vote; when this is the case, it seems important that social choice theorists are able to offer as complete a picture as possible of the advantages and disadvantages of each voting system to enable informed decisions. Furthermore, the profusion of associative structures, trade, foundations, Internet organizations overcoming the physical frontiers, offer a formidable testing ground that allows real human groups to benefit from the latest results of social choice and, eventually, to pass on their experience in this field to all forms of human organization seeking to improve their democratic functioning.










