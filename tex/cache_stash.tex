
\subsubsection{Consommation de la racine du \ac{DODAG}}

Dans le cas d'une racine, la puissance consommée pour transmettre à un nœud générique $i$ et recevoir sa réponse peut être exprimée par:

\begin{align}
  \Omega_{\racine, \tx} &= \frac{P_\tx T_{\req, \txtx} + P_\rx T_{\req, \txrx}}{r} 
  \label{contikimac:eq:racine_tx} \\
  \Omega_{\racine, \rx} &= \frac{P_\rx T_{\ans, \rxrx} + P_\tx T_{\ans, \rxtx}}{r} 
  \label{contikimac:eq:racine_rx}
\end{align}

Puisque la racine transmet à chacun de ses enfants, la puissance consommée pour transmettre à tous les nœuds et recevoir peut être exprimée par:

\begin{align}
  \Omega_{\racine, \tx } &= \sum_{i=1}^{N}\Omega_{\racine, \tx } 
    \label{contikimac:eq:root_tx} \\
    \Omega_{\racine, \rx } &= \sum_{i=1}^{N}\Omega_{\racine, \rx } 
    \label{contikimac:eq:root_rx}
\end{align}


\begin{align}
  \Omega_\sleep  &= \dfrac{T_\sleep P_\sleep}{T_\cycle} - \Gamma_\tx - \Gamma_\rx
  \label{contikimac:eq:omega_sleep} \\
  \Gamma_\tx &= \Omega_\sleep
  \frac{1}{T_\cycle}
  \left( \dfrac{3+\lfloor\frac{T_{\sleep_i}-T_\pkt}{T_\pkt}\rfloor}{2}(T_\pkt + T_\detect) + T_\ack \right)
  \label{contikimac:eq:gamma_tx}  \\
  \Gamma_\rx &= \Omega_\sleep
  \frac{1}{T_\cycle}
    \left( \frac{3 T_\pkt}{2} + T_\pkt + T_\ack \right)
    \label{contikimac:eq:gamma_rx}
\end{align}

où $P_\sleep$ est la puissance durant l'état de sommeil, $\Gamma_\tx$ et $\Gamma_\rx$ sont deux termes correctifs.

En temps normal, soit, un nœud fait une écoute du canal, soit transmet ou reçoit un paquet soit dort.
$\Gamma_\tx$ et $\Gamma_\rx$ sont utilisés pour raffiner la puissance consommée durant la phase de sommeil.
Les périodes de sommeils chevauchent celles de transmissions ou réceptions sur des temps courts ainsi sans ces termes la puissance consommée sera surestimée.

Le terme $\Gamma_\tx$ tient du fait que durant les opérations de transmissions comme les phases de
(i) strobbing d'un paquet sur une phase $T_\sleep$,
(ii) la transmission standard d'un paquet et
(iii) la réception d'un acquittement, un nœud serait normalement en sommeil.

Ainsi le facteur correctif $\Gamma_\tx$ est nécessaire puisque autrement l'énergie consommée par un nœud avec ce modèle serait surestimée car réception et transmission se chevaucherait avec les opérations de sommeil normales sur une période.
Des considérations similaires peuvent être dressées pour le terme $\Gamma_\rx$. Quand un nœud attends un acquittement pour transmettre un paquet \ac{ACK}, pour recevoir le préambule et le paquet, le nœud serait normalement en état de sommeil.

Lors du traitement de la requête, la phase de réception d'une trame entraîne d'une part l'utilisation de la radio en réception pendant $T_{\rxrx}$ mais d'autre part l'utilisation de la radio en transmission pour transmettre une réponse $T_{\rxtx}$.
% De même, la phase de transmission de la réponse entraîne  l'utilisation de la radio en transmission pendant $T_{\txtx}$ et en réception pendant $T_{\txrx}$.
