\chapter{Cache} % (fold)
\label{cha:cache}

\minitoc

\lipsum

\section{Introduction} % (fold)
\label{sec:cache_introduction}

\lipsum

% section introduction (end)

\section{État de l'art} % (fold)
\label{sec:_tat_de_l_art}

\lipsum

% section _tat_de_l_art (end)


\section{Cache Architecture} % (fold)
\label{sec:cache_architecture}

\lipsum

% section cache_architecture (end)

\section{Experiments} % (fold)
\label{sec:cache_experiments}

\lipsum

\subsection{CSMA} % (fold)
\label{sub:csma}

\lipsum

% subsection csma (end)

\subsection{TDMA} % (fold)
\label{sub:tdma}

\lipsum

% subsection tdma (end)

% section experiments (end)

\section{Performance Measurements} % (fold)
\label{sec:performance_measurements}

\subsection{Reactivity to network changes} % (fold)
\label{sub:reactivity_to_network_changes}

\begin{itemize}
	\item Combien de temps me faut il pour me rendre compte qu'un noeud est indisponible ?
	\item Combien de temps me faut il pour détecter une panne ?
	\item Comment les durée de cache s'adaptent a un changement de réseau ?
	\item Quel est le délai pour détecter un gros problème dans le réseau et adapter le niveau de requêtes histoire de calmer le jeu?
	\item Quel est le délai entre un changement dans le réseau significatif et un recalcul des temps de caches.
	\item Est ce que l'on peut avoir des phénomènes d'oscillation du système ? Typiquement des requêtes qui passent puis ne passent plus à la même fréquence. Comment lutter contre ce problème s'il existe?
	\item L'impact de la topologie se présente de manière évidente dans le calcul du temps de cache.

	\item est ce que ce système dans le cas ou il se plante bloque des utilisateurs comme 
	un ascenseur ou bien c'est plutôt un escalier de tel sorte qu'il ne bloque pas les utilisateurs d'en trouver un autre si il se plante.

	\item La fonction économique que l'on va utiliser peut être modifié selon différents critères. Par exemple on peut avoir une répartition des poids différentes selon qu'un noeud a plus ou moins de voisins. Une autre approche consiste a prendre en compte la répartition des popularités des requêtes envoyées.

	\item Est ce que ça fonctionne avec des topologies aléatoires ? Changeante (Ajout ou perte de noeuds)?

	\item Est ce que le modèle continue de fonctionner dans le cas ou on a plusieurs routeurs de bordure.	

\end{itemize}

% subsection reactivity_to_network_changes (end)

% section performance_measurements (end)

\section{Conclusion} % (fold)
\label{sec:cache_conclusion}

\lipsum

% section conclusion (end)

\section{Publications} % (fold)
\label{sec:cache_publications}

% section cache_publications (end)

% chapter cache (end)
