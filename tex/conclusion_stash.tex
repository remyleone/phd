\subsection{Passerelle virtualisée}

Les \ac{NFV} désignent une approche dans laquelle les fonctionnalités réseaux ne sont plus déployées sur du matériel spécialisé mais sur des ordinateurs classiques sous la forme de logiciels.
Dans le cas de la passerelle pour les \ac{LLN} il est toujours nécessaire de disposer de plusieurs liens radios différents.
Cependant les fonctionnalités de routage et de traduction applicatives sont logicielles.

Ainsi les fonctionnalités de supervision passive et de \ac{RPCA} peuvent également être utilisés dans ce contexte pour fournir les fonctionnalités de supervision et d'économies d'énergies tout en ne faisant aucune hypothèses sur le matériel de passerelle.


Pourquoi c'est utile: Réduction de cout, banalisation de la passerelle qui devient simplement une boite avec de multiples antennes qui est peuplée au cours du temps de manière dynamique.
Les exemples classiques de NFV incluent  load balancers, firewalls, intrusion detection device.
L'objectif est de réduire les couts et de ne pas avoir de materiel spécialisé.

La supervision passive et le \ac{RPCA} peuvent être considérés comme des \ac{NFV}, leur but consiste à apporter une fonctionnalité réseau supplémentaire.
Les \ac{NFV} utilise la virtualisation et les deux contributions peuvent être virtualisées afin d'être déployées n'importe où.
