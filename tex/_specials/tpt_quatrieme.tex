%\usepackage[left=1.3cm,top=0cm,right=1.3cm,bottom=1.2cm]{geometry}

% \usepackage{array}
% \usepackage{textcomp}
% \usepackage{helvet}	% or \usepackage{lmodern}
% \renewcommand\textnumero{n$^{\textsf{{\tiny O}}}$}
% \renewcommand{\familydefault}{\sfdefault}

\newgeometry{left=1.3cm,top=1cm,right=1.3cm,bottom=1.2cm}

\pagestyle{empty}

\AddToShipoutPicture*{\BackgroundPicLastPage}

\vspace{1.5cm}


\begin{center}{\LARGE \textbf{\myTitle}}\\
\vspace{.4cm}
{\large \textbf{\myName}}\\
\end{center}

\vspace{.9cm}

\textbf{RESUME :}

Les réseaux de capteurs sans fil (aussi appelés \ac{LLN}s en anglais) sont des réseaux contraints composés de nœuds ayant de faibles ressources (mémoire, CPU, batterie).
Ils sont de nature très hétérogène et utilisés dans des contextes variés comme la domotique ou les villes intelligentes.
Pour se connecter nativement à l'Internet, un \ac{LLN} utilise une passerelle, qui a une vue précise du trafic transitant entre Internet et le \ac{LLN} du fait de sa position.
Le but de cette thèse est d'exposer comment des fonctionnalités peuvent être ajoutées à une passerelle d'un \ac{LLN} dans le but d'optimiser l'utilisation des ressources limitées des nœuds contraints et d'améliorer la connaissance de leur état de fonctionnement.
La première contribution est un estimateur non intrusif utilisant le trafic passant par la passerelle pour inférer l'utilisation de la radio des nœuds contraints.
La seconde contribution adapte la durée de vie d’information mises en cache (afin d’utiliser les ressources en cache au lieu de solliciter le réseau) en fonction du compromis entre le coût et l'efficacité.
Enfin, la troisième contribution est Makesense, un framework permettant de documenter, d’exécuter et d’analyser une expérience pour réseaux de capteurs sans fil de façon reproductible à partir d'une description unique.

% \resumefr

\vspace{.6cm}

\textbf{MOTS-CLEFS:} Internet des objets, Réseaux de capteurs, Cache, Supervision, Recherche reproductible.

\vspace{1.0cm}

\textbf{ABSTRACT:}

\acf{LLN}s are constrained networks composed by nodes with little resources (memory, CPU, battery).
Those networks are typically used to provide real-time measurement of their environment in various contexts such as home automation or smart cities.
\ac{LLN}s connect to other networks by using a gateway that can host various enhancing features due to its key location between constrained and unconstrained devices.
This thesis shows three contributions aiming to improve the reliability and performance of a \ac{LLN} by using its gateway.
The first contribution provides a non-intrusive estimator of a node radio usage by observing its network traffic passing through the gateway.
The second contribution offers to determine the validity time of an information within a cache placed at the gateway to reduce the load on \ac{LLN}s nodes by doing a trade-off between energy cost and efficiency.
Finally, we provide Makesense, an open source framework for reproducible experiments that can document, execute and analyze a complete \ac{LLN} experiment on simulation or real nodes from a unique description.

\vspace{.6cm}
\textbf{KEY-WORDS:} Internet of Things, Low-Power and Lossy Networks, Cache, Monitoring, Reproducible Research

\restoregeometry
