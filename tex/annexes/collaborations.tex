\chapter{Collaborations extérieures}
\label{collaborations}

\section{A scalable and self-configuring architecture for service discovery in the internet of things}

Les déploiements visés par l'\ac{IoT} mettent en jeu des milliards d'appareils, ainsi ces appareils rendent de nouvelles formes d’interactions entre les personnes et les objets possibles.

Afin de permettre des applications robustes et faciles d'utilisation sur ces objets, il est nécessaire de disposer de mécanismes qui minimisent (ou annule) le besoin d'intervention humaine extérieure pour leur configuration et leur maintenance.
Ces mécanismes doivent pouvoir gérer un grand nombre d'objets qui sera en forte croissance au cours des prochaines années.

Cette contribution propose un mécanisme d'auto-configuration basée sur une architecture pair-à-pair, visant de grands réseaux d'objets et permettant d'obtenir un service de découverte de ressource automatisée ne nécessitant aucune intervention humaine pour leur configuration.
En particulier, cette contribution se concentre sur les mécanismes de découverte de services locaux et globaux en montrant comment l'architecture proposée permet une interaction tout en préservant une indépendance opérationnelle.

L'efficacité de l'architecture proposée a été confirmée par des résultats expérimentaux obtenus sur déploiement réel.

\subsection*{Publication}

Simone Cirani, Luca Davoli, Gianluigi Ferrari, R{\'e}my L{\'e}one, Paolo
  Medagliani, Marco Picone, and Luca Veltri.
\newblock A scalable and self-configuring architecture for service discovery in
  the internet of things.
  \newblock 2014.

\section{Bounding Degrees on RPL}

\ac{RPL} est le protocole de routage standardisé par l'\ac{IETF} optimisé pour les \ac{LLN}s.

La stabilité de \ac{RPL} peut souffrir de pertes de paquets et causer une instabilité des routes.
La plupart des solutions dédiées à ce problème se concentrent sur l'amélioration des métriques utilisées pour construire les routes.
Ces métriques sont le plus souvent basées sur une évaluation de la qualité du lien radio.

BD-RPL (Bounded Degree RPL) adopte une nouvelle approche en ajoute une contrainte supplémentaire sur le nombre maximal d'enfants qu'un routeur \ac{RPL} peut accepter lors de la construction des routes.
Cette méthode utilise les paquets de signalisation utilisés par défaut par \ac{RPL} et ne nécessite pas une nouvelle signalisation.

BD-RPL est évalué sur simulateur et implémenté sur des nœuds réels qui ont montré une amélioration de la fiabilité des transmissions de 10\%, une réduction de la consommation énergétique de 50\% et une amélioration de 60\% du délai.

\subsection*{Publication}

Fadwa Boubekeur, L{\'e}lia Blin, Remy Leone, and Paolo Medagliani.
\newblock Bounding degrees on rpl.
\newblock In {\em Proceedings of the 11th ACM Symposium on QoS and Security for
  Wireless and Mobile Networks}, pages 123--130. ACM, 2015.

% \section{Tactique de supervision active économe en énergie} 

% Collaboration avec Simon en suède sur le bandit manchot

% Mettre quelques graphes.